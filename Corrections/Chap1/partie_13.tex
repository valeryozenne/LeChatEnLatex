
\underline{\textbf{Résolution spatiale :}} L’acquisition en IRM nécessite un compromis entre le temps d’acquisition, la taille du voxel imagé (le nombre d’observables) et le rapport signal sur bruit obtenu. A temps équivalent, une résolution spatiale élevée va limiter la quantité de signaux mesurés et réduire drastiquement le signal sur bruit. La séquence d’acquisition doit fournir des images avec des résolutions spatiales suffisantes pour observer la zone traitée et caractériser le plus précisément possible la procédure interventionnelle.\\
