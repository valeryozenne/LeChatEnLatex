Dans un premier temps, ces outils ont été appliqués aux thérapies guidées par imagerie pour la cardiologie interventionnelle pour le traitement des arythmies cardiaques (cf chapitre \ref{chap2-ablation}). Pour évaluer progressivement la faisabilité des méthodes, des expériences ont été réalisées sur des échantillons test \textcolor{red}{(statiques et en mouvement)} puis sur des cœurs battants explantés \textcolor{red}{(présence du battement cardiaque, mais sans mouvements respiratoires)} et enfin sur des modèles précliniques. Les ablations par cathéter sont réalisées sur les oreillettes et les ventricules dans des conditions similaires à la pratique clinique. L’instrumentation \textcolor{red}{(filtres et générateurs)} permettant d’effectuer des ablations radiofréquence et des acquisitions IRM sans altération de la qualité d’image était déjà en place. Enfin, des validations cliniques ont été effectuées à l’Hôpital Haut-Lêveque de Bordeaux \textcolor{red}{(cf Paragraphe \ref{chap2-etude3-cliniqueventricule})}.\\