
La deuxième étape de la stratégie de correction de mouvements s’appuie sur les algorithmes de flots d’optiques qui sont couramment utilisés pour décrire le mouvement entre deux images. Les algorithmes de flots d’optiques analysent les variations d’intensité entre deux images successives pour estimer les champs déplacement\footnote{carte paramétrique indiquant les vecteurs de déplacement pour chaque voxel}, l’application de ces champs déplacement permet donc de corriger le mouvement observé. Ils sont particulièrement adaptés aux contraintes de temps réel et offrent une réponse rapide et robuste (cf paragraphe \ref{chap2-etude1-mrm} et paragraphe \ref{chap3-etude2-motion}).\\
