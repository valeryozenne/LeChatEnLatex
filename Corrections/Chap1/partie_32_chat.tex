Dans un premier temps, ces outils ont été appliqués aux thérapies guidées par imagerie pour la cardiologie interventionnelle pour le traitement des \textcolor{red}{arythmies} cardiaques (cf chapitre \ref{chap2-ablation}). Pour évaluer progressivement la faisabilité des méthodes, des expériences ont été réalisées sur des échantillons test (statique et en mouvement) puis sur des \textcolor{red}{cœurs} battants explantés (présence du battement cardiaque, mais sans mouvements respiratoires) et enfin sur des modèles précliniques. Les ablations par cathéter sont réalisées sur les oreillettes et les ventricules dans des conditions similaires à la pratique clinique. L’instrumentation (filtres et générateurs) permettant d’effectuer des ablations radiofréquence et des acquisitions IRM sans altération de la qualité d’image était déjà en place. Enfin, des validations cliniques ont été effectuées à l’Hôpital Haut-Lévêque de Bordeaux (cf \textcolor{red}{P}aragraphe \ref{chap2-etude3-cliniqueventricule}).\\