Le passage d’une chaîne de traitement linéaire à une chaîne de traitement asynchrone et interactive est un enjeu en imagerie IRM. Par exemple, la combinaison de séquences radiales avec des méthodes d’auto-synchronisation décrites au \textcolor{red}{paragraphe} précédent permettrait d’asservir le schéma d’acquisition. Cette méthode offre plusieurs avantages : une reconstruction évolutive (basse résolution puis haute résolution). Les données disponibles \textcolor{red}{pourront} être utilisées sur des instances différentes pour analyser le mouvement, segmenter des zones d’intérêt avec des méthodes d’apprentissage automatique, \textcolor{red}{calculer} des paramètres physiques. Cette approche multi-résolution a été récemment mise en œuvre pour les étapes de planification en radiothérapie par IRM-\ac{LINAC} \ref{bruijnen2019multiresolution}.