
L’IRM en temps réel, c’est-à-dire ayant spécifiquement des contraintes de temps réel (une chaîne de traitement correspondante est présentée en Figure \ref{fig-schema}, est utilisée pour l’imagerie fonctionnelle cérébrale à rétroaction, qui comprend de nombreuses applications : le traitement de troubles neurologiques (hyperactivité, épilepsie), le risque en chirurgie (via la vérification des fonctions cérébrales pré et post-opératoire) et l’exploration de la plasticité cérébrale. Des protocoles d’imagerie permettent à un individu de visualiser son activité cérébrale en temps réel. Soumis à des stimuli (visuel, sonore, mouvement), l’individu apprend à moduler l’activité de son cerveau par le biais de la mise en œuvre des stratégies mentales (en imaginant des tâches à effectuer par exemple), tout en ayant connaissance de son activité cérébrale. Pour avoir connaissance de son activité cérébrale, une séquence rapide d’\ac{IRMf} est dynamiquement acquise, et permet de cartographier le niveau d’oxygénation du sang\footnote{le signal BOLD, pour blood oxygenation level–dependent, qui est corrélé à l’activation du cerveau}. Après analyse des signaux, une grandeur physique décrivant l’activation du cerveau est transmise à l’individu (par exemple sous la forme d’un curseur). L’opérateur peut, en fonction des réponses aux stimuli, modifier les stratégies mentales que les patients doivent effectuer.\\
