Dans le cas \textcolor{red}{d}’un capteur interne à \textcolor{red}{l}’IRM, en se basant sur les séquences proposées ci-dessus, il est possible \textcolor{red}{d}’analyser les fréquences acquises \textcolor{red}{d}’obtenir de \textcolor{red}{l}’information sur les battements cardiaques et la respiration. Cette méthode est appelée auto-synchronisation (ou self-gating en anglais). L’extraction de \textcolor{red}{l}’intensité du point central de \textcolor{red}{l}’espace des k au cours du temps permet \textcolor{red}{d}’extraire à la fois une information respiratoire et une information cardiaque.\\