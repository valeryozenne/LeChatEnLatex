
Des capteurs externes à l’IRM peuvent aussi être mis en oeuvre. L’incorporation de micro-antennes de réception dans un cathéter permet déjà de localiser sa position dans l’aimant avec une précision de l’ordre du millimètre (cf paragraphe \ref{chap2-etude4-micro}, \ref{chap2-etude5-radial} et \ref{chap2-etude6-t1w}). Cette méthode de détection sera évaluée et comparée à la méthode d’autosynchronisation. Elle sera ensuite associée à un algorithme de reconstruction pour améliorer le tri des données acquises. Actuellement, le fonctionnement des techniques courantes de détection en imagerie IRM cardiaque est particulièrement sous-optimal\footnote{le navigateur 1D : c’est-à-dire une acquisition très rapide (30ms) tête pied lancé juste avant l’acquisition de chaque ligne de l’espace k qui permet d’observer par exemple la position du dôme du foie}. En effet, la plupart des données acquises sont exclues et perdues. Il serait plus judicieux de les intégrer à la reconstruction moyennant des corrections en mouvements pour gagner en signal sur bruit et diminuer par ailleurs le temps d’acquisition des séquences.\\
