
\underline{\textbf{Mouvement :}} Les méthodes développées doivent être robustes aux mouvements physiologiques (respiration, contraction cardiaque, flux sanguin dans les cavités,...) qui perturbent l’acquisition et induisent de nombreux artéfacts. Les mouvements intra-scans (temps d’acquisition d’une image) et inter-scans (entre deux acquisitions successives) doivent être minimisés pour permettre des mesures fiables dans le temps. Par ailleurs, les solutions choisies doivent être optimisées, car les méthodes de corrections de mouvements sont souvent coûteuses en temps de calcul. Les tâches les plus longues étant parfois les plus importantes, un travail d’optimisation algorithmique doit être effectué pour trouver le bon équilibre entre la précision et le temps nécessaire pour effectuer une tâche donnée.\\
