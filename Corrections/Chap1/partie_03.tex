
Dans la littérature, la définition de l’imagerie en temps réel en \ac{IRM} peut prêter à confusion \cite{dietz2018nomenclature}. En effet l’\ac{IRM} est perçue comme une modalité nécessitant un temps d’acquisition relativement long par opposition à la tomographie à rayons X ou à l’imagerie ultrasonore. L’\ac{IRM} en temps réel désigne parfois le fait d’acquérir continuellement des données pour obtenir un film représentatif d’un phénomène dynamique. Pour cela, la séquence d’acquisition est optimisée de manière à échantillonner le plus rapidement possible l’espace des k\footnote{ou espace de Fourier, espace dans le domaine fréquentiel où sont stockées les données brutes issues de l’acquisition avant reconstruction} et des algorithmes de reconstruction spécifiques (par exemple en utilisant une fenêtre glissante) sont utilisés pour reconstruire les images à un taux de rafraîchissement choisi. Ces méthodes permettent de visualiser des mouvements complexes (contraction cardiaque, mouvements de la langue lors de la parole, flux sanguin dans les grosses artères) mais ne constituent pas un système en temps réel en raison de l’absence même de contraintes de temps réel\footnote{par exemple la concordance des tâches, la régularité des mesures et de leur propagation, la réaction aux événements, la réaction aux pannes et aux conflits (notion de priorité)}.\\
