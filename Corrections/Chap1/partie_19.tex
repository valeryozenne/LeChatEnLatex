
Pour pallier à ces problèmes, je propose de développer plusieurs séquences, pour à terme combiner leurs avantages. La première se base sur l’acquisition simultanée (SMS pour Simultaneous Multi-Slice) de plusieurs coupes d’imagerie \cite{larkman2001investigation,setsompop2012blipped}. Ainsi, l’acquisition simultanée module les fréquences du signal IRM afin d’acquérir en une seule impulsion le signal de plusieurs coupes réparties dans l’espace (cf paragraphe \ref{chap3-etude1-sms}). Cela induit alors une superposition des signaux des coupes dans l’image. Grâce à des sensibilités des canaux de réception différents, les images peuvent être séparées avec les méthodes de reconstruction classique \cite{breuer2005controlled}. La seconde option consiste à imager la cible avec un champ de vue réduit (cf paragraphe \ref{chap2-etude6-t1w}). Soit en utilisant des séquences possédant un pulse d’excitation spatialement sélectif dans la direction de coupe et de phase \cite{saritas2008dwi}, soit avec des séquences dites SPEN (Spatiotemporal ENcoding) qui permettent de reconstruire une image sans utiliser la transformée de fourier dans la direction de phase \cite{ben2010super}. Ces deux méthodes permettraient d’éviter les repliements dans l’image et de réaliser de l’imagerie de température localisée.\\
