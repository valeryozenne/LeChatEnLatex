
\begin{description}
    \item C’est une modalité non irradiante (par opposition au scanner). 
    \item Elle bénéficie d’un excellent contraste aux tissus mous inégalé.
    \item Elle permet d’accéder à toutes les régions du corps (par opposition à l’échographie qui est
limitée par les interfaces osseuses).
    \item Elle bénéficie d’une souplesse d’utilisation pour le guidage (imagerie multi-plan et multidirectionnelle).
    \item Enfin elle permet de réaliser de l’imagerie en temps réel (notamment pour le guidage) .
    \item Elle permet de réaliser de l’imagerie fonctionnelle en cours d’intervention, notamment de
l’imagerie de température.
\end{description}

La maturité d’un point de vue clinique des méthodes d’IRM interventionnelle est très inégale : à titre d’exemple, le guidage d’aiguilles en neurochirurgie ou pour la chirurgie mini-invasive est relativement mature, tout comme la thermoablation par cryoablation en chirurgie abdominale. L’utilisation d’HIFU guidés par IRM pour des procédures mini-invasives d’ablation des tumeurs (prostate, sein et os) ou des tremblements essentiels connait un essor certain. Le guidage de cathéters sous IRM (mesures hémodynamiques, ablations radiofréquence cardiaque), bien que très prometteur, est quand à lui encore balbutiant.\\
