
Les travaux de recherche présentés ont pour objectif le développement de méthodes d’imagerie innovantes permettant de répondre aux enjeux et défis de l’IRM en temps réel dans le cadre de l’imagerie interventionnelle et notamment l’imagerie de température. Les organes visés sont le coeur et le foie. Les difficultés de l’imagerie IRM sur organes mobiles pour le diagnostic et les difficultés de l’imagerie IRM en temps réel interventionnelle pour la thérapie guidée sont donc réunies. Une brève description des verrous technologiques est listée ci-dessous.

\section{Principales difficultées scientifiques et verrous technologiques}
\label{chap1-verrous}

\underline{\textbf{Résolution temporelle :}} Les contraintes sur le temps d’acquisition sont un obstacle majeur pour l’acquisition en temps réel sur organes en mouvement rapide, du fait d’une cadence d’acquisition généralement trop faible avec les méthodes d’acquisition conventionnelles. La séquence d’acquisition doit être en mesure d’acquérir un volume en un minimum de temps (< 1s), afin de maintenir un taux de rafraichissement suffisant et permettre l’exploitation des données en temps réel.\\
