
L’ensemble de ces procédures impose des contraintes de temps réel très fortes, que ce soit en termes de vitesse d’acquisition et de reconstruction et de visualisation des données. Ces contraintes sont exacerbées lorsque la zone à traiter est en mouvement, notamment pour les organes abdominaux et le cœur. Ainsi, de nombreux verrous restent à lever et des solutions innovantes doivent être développées pour faire de l’IRM l’outil de référence dans le domaine de la thérapie guidée par l’image.

\section{La thermométrie IRM}
\label{chap1-thermometrie}

\subsection{Mesure de champ thermique par IRM, les concepts clés}

L'imagerie de température par IRM est une méthode de mesure thermique non-invasive permettant de cartographier l'élévation relative de la température au sein des organes au cours de la procédure d'ablation. La mesure repose sur l'acquisition consécutive de plusieurs images au cours de la procédure et permet de visualiser à la fois la distribution spatiale (en 2D ou en 3D) de la température à un instant t\footnote{enfin plutôt sur une durée $\Delta$t} et la cinétique (ou l'évolution temporelle) de la température. Avant de présenter la méthode de mesure, il convient de rappeler que la mesure IRM n'est pas une mesure ponctuelle. La mesure de la température est effectuée d'une part sur un élement de volume appelé voxel et correspond à la moyenne de la température sur cet élément de volume (on parle de volume partiel). D'autre part le temps d'acquisition n'est pas instantané (par rapport aux variations temporelles de température). La mesure IRM est moyennée temporellement sur le durée d'acquisition. \\
