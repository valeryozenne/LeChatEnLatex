Enfin, les méthodes d’encodage de l’espace des \textcolor{red}{$k$} non cartésiennes comme les méthodes radiales ou spirales \cite{wright20143d} peuvent être fortement sous-échantillonnées pour accélérer l’acquisition (cf paragraphe \ref{chap2-etude5-radial}). Elles \textcolor{red}{présentent} l’avantage d’échantillonner le centre de l’espace des \textcolor{red}{$k$} ce qui contribue à la fois aux hautes et basses fréquences de l’image et l’exploitation du point central de l’espace des \textcolor{red}{$k$} contient des informations sur les mouvements physiologiques. Néanmoins, la plupart de ces méthodes de reconstruction associées dites itératives, \cite{lustig2007sparse} ne sont cependant pas assez efficaces ou non adaptées aux contraintes de temps réel de l’imagerie interventionnelle (temps de calcul qui dépasse parfois plusieurs heures) mais elles présentent un intérêt certain pour imager les organes cibles dans chaque état physiologique (imagerie 4D/5D).

\subsection{Développement de nouvelles stratégies permettant de s'affranchir du mouvement}

Les mouvements combinés de respiration et de battement cardiaque représentent un défi technique pour l’imagerie en temps réel. Pour s’affranchir de la contraction cardiaque, les acquisitions peuvent être synchronisées avec le signal \ac{ECG}. Il en est de même avec le mouvement respiratoire où les acquisitions peuvent être synchronisées sur la phase la plus stable de la respiration. Cependant, ces méthodes finissent par être \textcolor{red}{coûteuses} en temps et ne sont pas toujours bien adaptées (la synchronisation ECG fonctionne mal en présence d’arythmie, cf paragraphe \ref{chap2-etude3-cliniqueventricule}). De nouvelles approches seront donc proposées pour s’affranchir du mouvement ou le corriger en temps réel. Nous pouvons distinguer les méthodes de détection (qui permettent de trier les données) et les méthodes de correction (qui permettent de recaler les images). La première étape de la stratégie que je propose consiste à décrire le mouvement à l’aide d'un capteur en temps réel. Ce capteur peut être soit interne soit externe à l’IRM.