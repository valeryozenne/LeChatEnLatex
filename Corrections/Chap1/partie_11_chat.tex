Cette relation théorique \textcolor{red}{a été} validée à de nombreuses reprises et est utilisée dans des régions très sensibles telles que le thalamus pour le traitement des tremblements essentiels \cite{huang2018predicting}. Elle reste néanmoins sujette à caution \cite{o2012estimation} notamment dans les organes \textcolor{red}{extrêmement} perfusés. Le caractère exponentiel la rend par ailleurs très sensible au bruit de mesure.

\subsection{De la séquence à la procédure}

La technologie d'\textcolor{red}{imagerie de température} par IRM, \textcolor{red}{ce qui inclut} la séquence d'acquisition et la méthode de reconstruction, est indissociable de la procédure dans laquelle elle est utilisée. Les caractéristiques de la séquence doivent être définies en fonction de la durée d'ablation, de la quantité d'énergie déposée par seconde. D'autre part, ces contraintes sont exacerbées en présence de boucles de rétroaction pour piloter la position ou la puissance des dispositifs médicaux lors de la procédure \cite{bour2018real,desclides2023real, ozenne2016automatic}. Enfin, la procédure interventionnelle peut être relativement complexe à mettre en œuvre et nécessite donc des méthodes IRM simples et fiables.\\