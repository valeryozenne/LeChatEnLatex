
\begin{description}
\item  Le guidage d’aiguilles par voie percutanée : infiltrations, biopsies, drainages et aspirations.
\item Les thermoablations : cryoablations, laser, micro-ondes, radiofréquence, \ac{HIFU} de tissus pathologiques.
\end{description}


%\begin{figure}[hbtp]
	%\begin{bigcenter}
%       \includegraphics[width=1.0\linewidth]{images/Chapitre1/schema_general.png}
%        \caption[]{\textbf{Chaîne de traitement IRM en temps réel.} qui comprend acquisition, reconstruction d’images, correction de mouvements, analyse et visualisation des données. L’opérateur peut interagir directement ou indirectement avec le dispositif médical. Suivant la procédure, la chaîne de traitement est utilisée pour retransmettre des informations en temps réel (par exemple le signal d’activation du cerveau), piloter le dispositif en temps réel (par exemple guidage d’aiguilles) ou assurer la sécurité du patient en ayant un rétro-contrôle bloquant pendant la procédure.} 
%        \label{fig-schema}
%       % \end{bigcenter}
%\end{figure}

Dans ces deux cas, malgré la présence de contraintes liées à la présence du champ magnétique, l’IRM offre des avantages significatifs :\\
