Ce savoir-faire est \textcolor{red}{essentiel} et mis à contribution dans la majorité des études (cf chapitre \ref{chap2-ablation}, chapitre \ref{chap3-oncologie} et Annexe \ref{chap-annexe3}). La mise en place d’une telle infrastructure est associée à trois caractéristiques :
\begin{description}
\item Modulaire : l’environnement est ajustable en fonction des besoins
\item Immédiat : les développements réalisés sont visibles instantanément par le chercheur (lors du développement), le praticien (lors des tests de faisabilité)
\item Portable : les choix technologiques permettent un transfert clinique immédiat
\end{description}

\subsection{Validation préclinique et clinique et conclusion}
\label{chap4-conclusion}
%sont axés sur le développement de l’imagerie IRM en temps réel dans le but de

En conclusion, les travaux de recherche présentés visent à guider des interventions thérapeutiques sur organes mobiles. Pour mener à bien ce projet, il a été nécessaire d’imaginer des méthodes d’acquisitions originales, couplées à des stratégies robustes de corrections de mouvements. Tous ces développements ont été intégrés dans un environnement logiciel flexible et performant pour permettre le suivi des procédures en temps réel.