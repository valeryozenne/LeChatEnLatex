
\subsection{Développement de l’IRM en temps réel adaptative et interactive}

L’IRM adaptative vise à modifier en temps réel le schéma d’acquisition soit dans l’espace réel, soit dans l’espace des k. Le premier cas permet par exemple d’asservir la position des coupes en fonction d’un capteur qui peut être interne à l’IRM (le navigateur 1D) ou externe (la position de l’extrémité d’un cathéter, le signal d’un appareil échographique, etc.).\\
