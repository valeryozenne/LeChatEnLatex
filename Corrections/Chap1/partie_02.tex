\acro{ACP}{analyse en composante principale}
\acro{ANR}{Agence Nationale de la Recherche}
\acro{CARTLOVE}{CARTLOVE}
\acro{CEM43}{dose cumulée équivalente à 43°C ou de l'anglais Cumulated Equivalent Minute}
\acro{CHAUPATAT}{CHAUPATAT}
\acro{CHC}{Carcinome Hépatocellulaire}
\acro{CIFRE}{convention industrielle de formation par la recherche}
\acro{CNN}{réseau de neurones à convolution ou de l'anglais convolutional neural network }
\acro{CPU}{processeur ou de l'anglais central processing unit}
\acro{CNRS}{centre national de la recherche scientifique}
\acro{CRMSB}{Centre de Résonance Magnétique des Systèmes Biologiques}
\acro{CS}{acquisition comprimée ou de l'anglais compressed sensing}
\acro{CSD}{modèle de déconvolution sphérique contrainte}
\acro{DFG}{fondation allemande pour la recherche ou en allemand Deutsche Forschungsgemeinschaft}
\acro{DICOM}{de l'anglais Digital imaging and communications in medicine}
\acro{DTI}{imagerie du tenseur de diffusion ou de l'anglais Diffusion tensor Imaging}
\acro{ECG}[ECG]{électrocardiogramme}
\acro{EPI}{imagerie echo planaire ou de l'anglais Echo-Planar Imaging}
\acro{FLASH}{Fast low angle shot magnetic resonance imaging}
\acro{GRAPPA}{de l'anglais GeneRalized Autocalibrating Partial Parallel Acquisition}
\acro{GRE}{echo de gradient ou de l'anglais Gradient echo}
\acro{GPU}[GPU]{processeur graphique, ou de l'anglais Graphics Processing Unit}
\acro{HIFU}{ultrasons focalisés de haute intensité ou de l'anglais High Focus Ultrasound}
\acro{I2M}{institut de mécanique et d'ingénierie}
\acro{IHU}[IHU]{institut hospitalo-universitaire}
\acro{IRM}{Imagerie par Résonance Magnétique ou de l'anglais Magnetic Resonance Imaging}
\acro{IRMf}{imagerie fonctionnelle}
\acro{LINAC}{linear particle accelerator}
\acro{Liryc}{des maladies du rythme cardiaque}
\acro{LITT}{laser ou Laser-induced interstitial thermotherapy}
\acro{LMU}{Ludwig Maximilian University}
\acro{PDF}{projection sur dipole magnétique ou de l'anglais Projection into field}
\acro{MB}{multibande}
\acro{MRS}{Magnetic Resonance Spectrometry}
\acro{OF}{flux d'optique ou Optical Flow}
\acro{ppm}{partie par million}
\acro{PRFs}{déplacement de la fréquence de résonance du proton ou de l'anglais Proton Resonance Frequency Shift}
\acro{RF}{radiofréquence}
\acro{QSM}{imagerie de susceptibilité magnétique quantitative ou de l'anglais Quantitative Susceptibilité Mapping}
\acro{SMARTHEAT}{Spatial Mapping and Analysis of Real-time MRI Thermometry data for Highly Efficient liver tumor Ablation using inverse Thermal modeling}
\acro{SMS}{multicoupes simultanées ou de l'anglais Simultaneous Multi Slice}
\acro{STI}{imagerie du tenseur de structure ou de l'anglais Structure tensor Imaging}
\acro{TIFC}{Imagerie et Caractérisation Thermique ou Thermal Imaging Field Characterization}
\acro{TDM}{tomodensitométrie}
\acro{TDMCP}{tomodensitométrie à comptage photonique}
\acro{UNMASC}{UNMASC}

\end{acronym}
%!TEX root = Manuscrit.tex

%\chapter{Introduction}
\label{chap1-intro}
\minitoc

\section{La thérapie guidée par l'image}

Le temps réel en traitement d’image désignent un processus pour lequel le taux de rafraîchissement des image est plus rapide que l’évolution du processus observé. Plus généralement, un système en temps réel nécessite la capacité de contrôlé un procédé à une vitesse adaptée à l’évolution du procédé contrôlé. Les opérations effectués nécessitent alors des garanties afin quelles soient complétées dans les délais impartis.\\
