
\subsection{Intégration des méthodes dans un environnement souples et en temps réel}

En imagerie en temps réel, la cadence d’acquisition et la quantité de flux de données (de 10 à 50 Mo/s) impose au reconstructeur d’image de la machine IRM (fournie par le constructeur) des contraintes non prévues dans son cahier des charges. Pour cela, il est nécessaire de mettre en place une infrastructure réseau beaucoup plus performante permettant de dériver l’acquisition des données en temps réel pour effectuer la reconstruction sur un calculateur externe. Lors de mon stage post-doctoral j’ai activement contribué à la mise en place d’un tel environnement de travail sur plusieurs sites de recherche et sites cliniques via l’installation du \GADGETRON\footnote{le \GADGETRON est un logiciel de reconstruction d’image IRM open-source}. Le \GADGETRON offre une souplesse d’utilisation inégalée que ce soit pour le prototypage, le développement ou le déploiement de solutions en environnement clinique. L’aspect open-source et indépendant du constructeur permet de dimensionner l’environnement de travail suivant ces besoins : le système choisi peut-être un ordinateur, un réseau local d’ordinateurs, un calculateur national, un calculateur privé (Azure, Amazon Cloud, ...), le tout directement connecté à l’IRM. Le \GADGETRON dispose d’un grand nombre de librairies optimisées pour l’imagerie en temps réel, l’utilisateur à la possibilité de compléter ou créer ces propres librairies dans des langages scientifiques (\MATLAB, Python) ou adaptés aux contraintes de temps réel (C++, \CUDA). Les possibilités sont immenses, à titre d’exemple les dernières avancées de l’équipe de développement intégre le logiciel TensorFlow \footnote{un outil open source d’apprentissage automatique développé par Google} qui est directement accessible depuis l’IRM. Le cadre de nos développements est aussi ambitieux, l’ensemble des solutions présentées dans ce projet est intégré au \GADGETRON, le guidage où le suivi en température nécessitent des algorithmes performants et extrêmement robustes. A titre d’exemple, les séquences de thermométrie nécessitent des temps de calculs de l’ordre de la dizaine de millisecondes par coupe pour préserver un taux de rafraîchissement inférieur à la seconde. Par ailleurs, le \GADGETRON intégre des solutions de containérisation d’applications informatique qui facilitera grandement le déploiement de nos développements méthodologiques sur plusieurs sites lors d’études multi-centriques\cite{de2024fully,de2024advanced}.\\
